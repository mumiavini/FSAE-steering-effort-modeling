\documentclass[a4paper]{report}

% --- Configurações Básicas ---
\usepackage[T1]{fontenc}      % Codificação de saída da fonte (Importante!)
\usepackage[utf8]{inputenc}   % CORREÇÃO: É 'utf8' (sem hífen)
\usepackage[english]{babel}   % Idioma do documento (para hifenização correta)
\usepackage[margin=1in]{geometry}
\usepackage{graphicx}
\usepackage{amsmath}
\usepackage{hyperref}
\usepackage{fancyhdr}
\usepackage{tocloft}
\usepackage{parskip}          % DICA: Remove indentação e adiciona espaço entre parágrafos (comum em manuais de engenharia)

% --- Metadados ---
\title{Steering Effort Calculation and Validation Handbook}
\author{FSAE26 Team}
\date{\today}

% --- Cabeçalho e Rodapé ---
\pagestyle{fancy}
\fancyhf{} % Limpa configurações padrão
\fancyhead[L]{Steering Effort Handbook}
\fancyhead[R]{FSAE26}
\fancyfoot[C]{\thepage} % Numeração de página no centro do rodapé

\begin{document}

\maketitle

\tableofcontents
% O comando \newpage não é estritamente necessário aqui pois \chapter já quebra página, 
% mas não faz mal mantê-lo se quiser garantir uma página em branco extra.

\chapter{Introduction}
\section{Purpose}
This handbook documents the steering effort calculation and validation procedures for the FSAE26 vehicle. It provides comprehensive guidelines for analyzing steering system performance and ensuring design requirements are met.

\section{Scope}
\begin{itemize}
    \item Steering system analysis methodology
    \item Calculation procedures and formulas
    \item Validation testing protocols
    \item Performance benchmarks
\end{itemize}

\chapter{Steering System Overview}
\section{System Components}
Describe the steering system architecture and key components.

\section{Design Requirements}
Outline the steering effort targets and constraints.

\chapter{Calculation Methodology}
\section{Theory and Formulas}
Document the mathematical foundations for steering effort calculations. 
% Exemplo de como você pode inserir uma fórmula relevante para direção:
% Align torque equation example
\begin{equation}
    M_z = F_y \cdot (n_m + n_p) \cdot \cos(\lambda)
\end{equation}

\section{Input Parameters}
List required inputs and measurement procedures.

\chapter{Validation Procedures}
\section{Testing Protocol}
Define validation testing methods and acceptance criteria.

\chapter{Results and Analysis}
\section{Data Presentation}
Present calculation results and validation findings.

\end{document}